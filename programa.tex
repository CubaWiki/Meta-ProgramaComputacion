\documentclass[spanish]{article}

%M\'argenes piolas
\addtolength{\oddsidemargin}{-.5in}
\addtolength{\evensidemargin}{-.5in}
\addtolength{\textwidth}{1in}
\addtolength{\topmargin}{-.5in}
\addtolength{\textheight}{1in}

\usepackage[T1]{fontenc}

\usepackage{babel}
\usepackage[utf8]{inputenc}
\usepackage{amsfonts}
\usepackage{amsmath,amssymb}
\usepackage[shortlabels]{enumitem}
\usepackage{mathrsfs}
\usepackage{hyperref}
\hypersetup{
      colorlinks,
      citecolor=black,
      filecolor=black,
      linkcolor=black,
      urlcolor=black
}
\usepackage{xcolor}

%no section numbering
\setcounter{secnumdepth}{0}

\newcommand{\RR}{\mathbb{R}}


%% README:
%% Traduccion de lo programas de las materias de Matematica y Computacion de la UBA
%% Es probable que esto contenga errores. Tambien falta calculo numerico para matematicos.

\begin{document}

\section{General remarks}
\begin{itemize}
  \item
  The recommended bibliography of the courses usually contains four or five books
  but I include only the ones that I consulted during the course.

  \item
  ``Chapters $x$ to $y$'' means that chapter $y$ is included.

  \item
  Computer Science courses usually require one or more projects in which the student
  applies techniques and implements the algorithms studied in the course.

  \item
  All the Computer Science courses offer course notes.

  \item
  The outlines of Computer Science courses are much more succinct, since this is
  how they are presented in the official study plans.

  \item
  I had to manually translate the outlines, so in this sense they are not
  official.
\end{itemize}

\section{Department of Computer Science -- required courses}

\subsection{Probability and Statistics [Computer Science]}
No projects.

\subsubsection{Objectives}
Introduce the basic concepts in probability theory and statistics

\subsubsection{Contents}
Descriptive statistics. Probability spaces. Discrete random variables.
Continuous random variables. Generation of random number. Elemental statistics.
Applications.

\subsubsection{Bibliography}
Subsumed by the bibliography of Probability and Statistics [Mathematics].


\hrulefill%------------------------------

\subsection{Numerical Methods}
Two group projects.

\subsubsection{Objectives}
Give the basic tools for the treatment of numeric problems. Study
the most important numerical methods and their computational implementations.

\subsubsection{Contents}
Rounding error. Numerical representations. Solving non linear systems.
Interpolation and polynomial approximation. Least squares. Numerical
differentiation and integration. Ordinary differential equations. Solving
linear systems: direct methods and iterative methods. Eigenvalues and eigenvectors.

\subsubsection{Bibliography}
\begin{itemize}
  \item Watkins. Fundamentals of matrix computations.\\
    Complete.
  \item Course notes.
\end{itemize}


\hrulefill%------------------------------

\subsection{Algorithms and Data Structures I}
Three group projects.

\subsubsection{Objectives}
Present tools to solve problems about sequences and lists. Formally prove
the correctness of programs. Make small projects to apply the learned
tools.

\subsubsection{Contents}
Specification and implementation of programs. Correctness of programs. Types.
Abstract types. Treatment of sequences and lists. Sequential files.

\subsubsection{Bibliography}
\begin{itemize}
  \item Course notes.
\end{itemize}


\hrulefill%------------------------------

\subsection{Algorithms and Data Structures II}
Three group projects.

\subsubsection{Objectives}
Introduce abstract recursive data types. Present specification tools.
Present techniques of analysis and design of algorithms. Proof the correctness
of the constructed programs. Implement the learned techniques in projects.
Analyze time and space complexity of various algorithms.

\subsubsection{Contents}
Recursion. Examples of abstract types, lists, queues, dictionaries,
trees, graphs,\ldots. Methodologies of formal specification.
Formal specifications and implementation of these types using various
data structures. Basics on time and space complexity.

\subsubsection{Bibliography}
\begin{itemize}
  \item Cormen, Leiserson, Rivest and Stein. Introduction to Algorithms. 3 ed.\\
    Chapters $1$ to $5$ of Part I. Parts II and III.
  \item Course notes.
\end{itemize}


\hrulefill%------------------------------

\subsection{Algorithms and Data Structures III}
Three group projects.

\subsubsection{Objectives}
Introduce the notion of graph. Implement using graphs various types
of abstraction. Present the notion of exponential time algorithms,
approximate solutions and various heuristics. Apply the learned concepts
in various projects.

\subsubsection{Contents}
Graphs. Basic topics in graph theory. Solving problems using graphs.
Basic topics in complexity theory. Recurrence. Heuristic solutions.

\subsubsection{Bibliography}
\begin{itemize}
  \item Cormen, Leiserson, Rivest and Stein. Introduction to Algorithms. 3 ed.\\
    Parts IV and VI.
  \item Course notes.
\end{itemize}


\hrulefill%------------------------------

\subsection{Paradigms of Programming Languages}
Three group projects.

\subsubsection{Objectives}
Evaluate the concepts of programming language in terms of its contribution
to the developing of software.

\subsubsection{Contents}
The imperative paradigm. The functional paradigm. The logic paradigm.
The equational paradigm. Elements for the evaluation of languages.

\subsubsection{Bibliography}
\begin{itemize}
  \item Pierce. Types and Programming Languages.
    Part II. And chapter $15$ of Part III.
  \item Lloyd. Foundations of Logic Programming.
    Chapters $1$ and $2$.
  \item Course notes.
\end{itemize}


\hrulefill%------------------------------

\subsection{Computer Architecture I}
Two group projects.

\subsubsection{Objectives}
Understand the components in a computational system, their interaction and
operation, abstracted by Von Neumann.

\subsubsection{Contents}
Representation of information. Components in a classical computational system.
Languages. Microprogramming.

\subsubsection{Bibliography}
\begin{itemize}
  \item Stallings. Computer Organization and Architecture.\\
    Part Two.
  \item Course notes.
\end{itemize}


\hrulefill%------------------------------

\subsection{Computer Architecture II}
One individual project and two group projects.

\subsubsection{Objectives}
Develop projects to experiment with various computational systems.

\subsubsection{Contents}
Programming using microinstructions. Program interactions between various
components of the system. Study a particular implementation of assembly language.
Program a microkernell using INTEL assembly and C.

\subsubsection{Bibliography}
\begin{itemize}
  \item Course notes.
  \item Tom Shamley. The Unabridged Pentium $4$. IA32 Processor Genealogy, MindShare, INC. Addison-Wesley.\\
    Used as a reference book.
\end{itemize}


\hrulefill%------------------------------

\subsection{Operating Systems}
Three group projects.

\subsubsection{Objectives}
Introduce the main functionalities of an operating system. Present the interrelation
between operating system and the computer architecture.

\subsubsection{Contents}
Operating Systems. Components and functions. Administration of the various components
of the system. Concurrent and distributed processes.

\subsubsection{Bibliography}
\begin{itemize}
  \item Tanenbaum. Modern Operating Systems.\\
    Chapters $1$ to $6$ and $9$.
  \item Course notes.
\end{itemize}


\hrulefill%------------------------------

\subsection{Communication Theory}
Three group projects.

\subsubsection{Objectives}
Give the knowledge needed to understand the principles of transmission
and manipulation of information. Study the design of information networks.

\subsubsection{Contents}
Information theory. Coding of information. Transmission mediums. Bandwidth and
capacity. Fourier analysis. Modems. Multiplexers. Queue Theory. Protocols.

\subsubsection{Bibliography}
\begin{itemize}
  \item Peterson. Computer Networks: A System Approach.\\
    Complete.
  \item Course notes.
\end{itemize}


\hrulefill%------------------------------

\subsection{Language Theory}
One group project.

\subsubsection{Objectives}
Present the notion of formal languages, syntax and semantics of languages,
with compilers in mind.

\subsubsection{Contents}
Formal languages. Syntax and semantic of programming languages. Construction
of some simple compilers.

\subsubsection{Bibliography}
\begin{itemize}
  \item Hopcroft, Ullman. Introduction to Automata Theory, Languages, and
    Computation.\\
    Chapters $1$ to $7$.
  \item Aho, Sethi, Ullman. Compilers: Principles, Techniques, and Tools.\\
    Used as a reference book for one of the projects.
  \item Course notes.
\end{itemize}


\hrulefill%------------------------------

\subsection{Software Engineering I}
Two group projects.

\subsubsection{Objectives}
Present and practice traditional techniques of Software Engineering and
design, to be able to cope with complex systems.

\subsubsection{Contents}
Planning of software projects. Requirements analysis. Specification.
Design. Studying software quality: correctness, trustworthiness.
Strategies of software verification. Maintenance. Metrics. Cost computation.
CASE tools.

\subsubsection{Bibliography}
\begin{itemize}
  \item Axel van Lamsweerde. Requirements Engineering: From System Goals to UML Models to
    Software Specifications. Wiley $2009$.\\
    Chapters $1$, $7$, $8$. Chapters $10$ to $17$.
  \item Course notes.
\end{itemize}


\hrulefill%------------------------------

\subsection{Software Engineering II}
Two group projects.

\subsubsection{Objectives}
Present and practice traditional techniques of software engineering and
design, to be able to cope with complex systems.

\subsubsection{Contents}
Techniques of object oriented software engineering. Fast prototyping.
Exercise the planning and management of software projects.

\subsubsection{Bibliography}
\begin{itemize}
  \item Gamma, Helm, Johnson, Vlissides. Design Patterns - Elements of Reusable
    Objet-Oriented Software.\\
    Chapter $1$. Chapters $3$ to $6$.
  \item Course notes.
\end{itemize}


\hrulefill%------------------------------

\subsection{Databases}
Three group projects.

\subsubsection{Objectives}
Extend the concept of data structure to the requirements involved in the solving
of complex problems.

\subsubsection{Contents}
Functionalities of Database systems. Models of data. Query languages.
Design of Databases. Physical data structures. Query optimization. Concurrence
and recovering. Implementations.

\subsubsection{Bibliography}
\begin{itemize}
  \item Ullman. Principles of Database and Knowledge Base Systems.\\
  \item Course notes.

\end{itemize}

\hrulefill%------------------------------

\section{Department of Computer Science -- elective courses}

\subsection{Modal Logic}
No projects.

\begin{itemize}
  \item Formulas. Induction principle. Recursive definitions. Subformulas. Substitution.
  \item Classical, regular and normal logics. Axiomatizable and recursively axiomatizable
    logics. Deductibility. Consistence. The K logic. Tautologic consequence.
  \item Frames. Models. The logic of a class of frames. Completeness. Correctness. Subframes
    and submodels generated by indexes. Strong and weak consequence.
  \item Properties of relations. Correspondence between modal formulas and properties of
    relations. Correspondence between modal formulas and first and second order formulas.
  \item Sets of consistent formulas and L-consistent formulas. Canonical models.
    Completeness of modal logics. Existence theorems of models and frames.
  \item The property of finite models and the property of finite frames. Decidability from
    the property of finite frames.
\end{itemize}

\subsubsection{Bibliography}
\begin{itemize}
  \item Ram\'on Jansana. Una Introduccion a la L\'ogica Modal.\\
    Chapters $1$ to $7$.
\end{itemize}


\hrulefill%------------------------------

\subsection{Positional Games}
No projects.

\begin{itemize}
  \item
Positional games, examples. Different sets of rules. Tic-Tac-Toe, generalizations. Hex.
  \item
Strong games, pairing draw. Maker-Breaker games. Biased games, threshold bias. General tools.
  \item
Standard games on graphs – Connectivity game, Degree game, Hamilton Cycle game. Games with non-spanning winning sets, Clique game, Planarity game, Colorability game.
  \item
Avoider-Enforcer games, two variants of rules, threshold biases. Standard games on graphs.
  \item
Winning fast, implications in strong games.
  \item
Games on sparse graphs, variations. Games on random boards.
  \item
Neighborhood conjecture and related problems.
\end{itemize}

\subsubsection{Bibliography}
\begin{itemize}
  \item Handouts that were parts of the book: Hefetz, Krivelevich, Stojakovic, Szabo. Positional Games.\\
    Parts of chapters $1$ to $4$.
\end{itemize}

\hrulefill%------------------------------

\subsection{Interactive Theorem Proving: theory and practice}
One group project.

\begin{itemize}
  \item
    Brief presentation of Proof Theory and Coq.

  \item
    Propositional Calculus based on natural deduction. Rules, examples and basic theorems. Proving
    propositional theorems in Coq.

  \item
    Lambda Calculus. Curry-Howard correspondence. Inspection of proof terms.

  \item
    First Order Calculus based on natural deduction. Rules, examples and basic theorems.
    Proving first order theorems in Coq.

  \item
    Inductive Types and structural induction.

  \item
    Fixed points and its relation to structural induction.

  \item
    Lambda Calculus with fixed points. Non termination and inconsistency.

  \item
    Proof automation: proof search, Ltac, typed automation.

\end{itemize}

\subsubsection{Bibliography}
\begin{itemize}
  \item Y. Bertot and P. Cast\'eran: Interactive Theorem Proving and Program Development, Springer, 1998.\\
    Chapters $1$ to $4$ and $6$ to $8$.

  \item Benjamin C. Pierce. Software Foundations.\\
    In the lab we made exercises from this book. Chapters: Basics, Induction, Lists, MoreCoq.

  \item Course handouts.

\end{itemize}
\hrulefill%------------------------------

\subsection{Operations Research}
One individual project.

\subsubsection{Objectives}
Lots problems in Combinatorial Optimization can be modeled as lineal programming problems.
Many of these problems are difficult to solve exactly.
The purpose of this course is learn to formulate models and study the methods
for solving a big variety of optimization problems that can be modeled as
integer and linear programming problems.

\subsubsection{Contents}
\begin{itemize}
    \item Linear Programming.
    \item Convex sets and functions. Polyhedra and cones. Convex hull. Basic lemmas and theorems.
    \item Simplex Method. Convergence. Complexity. Dual problem. Dual theorems.
    \item Examples of Integer Programming problems. Good and bad formulations.
    \item Characterizations of the convex hull of an Integer Programming problem. Classical inequalities.
    \item Algorithms for the resolution of Integer Programming problems. Branch and Bound. Branch and Cut.
      Usage and experimentation with CPLEX. Small project in CPLEX.
\end{itemize}

\subsubsection{Bibliography}
\begin{itemize}
  \item Chvatal. Linear Programming.\\
    Part I.
  \item Nemhauser. Integer and Combinatorial Optimization.\\
    Used as a reference book.
\end{itemize}



\end{document}
